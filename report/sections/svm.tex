Support Vector Machines are supervised maximum margin models that are used for problems such as:
\begin{itemize}
    \item Classification
    \item Regression
    \item Distribution Estimation
\end{itemize}
They are considered supervised because they need a pre-labeled training set to develop 
the hyperplanes that separate the classes. The terminology maximum margin refers to the
fact that the SVM tries to solve a convex optimization problem which maximizes the 
distance between the hyperplane and the closest data points from each class. This 
distance is called "margin" and is mathematically defined as:
\begin{equation}
    \max_{w, b}{\|w\|^{-2}}
\end{equation}
\begin{equation}
    w^Tx_i + b \geq 1 \quad \forall x_i \in C_1
\end{equation}
\begin{equation}
    w^Tx_j + b \leq 1 \quad \forall x_j \in C_2
\end{equation}
